\section{\label{sec:setup:incompleteness} Finite and infinite incompleteness}

The common theme of this thesis is addressing \textit{incompleteness} in training and evaluation datasets with on-demand human feedback. 
In this section, we will expand on the intuition we presented in \refchap{intro} to provide a more formal description of what incompleteness is.

Intuitively, incompleteness arises because the data we are able to observe (the test collection)  does not capture phenomena in the data we wish to evaluate (the system's predictions).
Formally,
  let $\sX$ be our universe of objects (e.g.\ relational triples in KBP, answers to questions or summaries of articles),
  let $X \subset \sX$ be the observable data
  and let $Y \subset \sX$ be the target data that we wish to evaluate.
Our goal is to measure some aggregate property $f$ (e.g.\
accuracy, precision, etc.) of $Y$, $\mu_f = \E_{p(Y)}[f(y)] \eqdef \sum_{y \in Y} p(y) f(y)$, where $p(y)$ is a given distribution or measure over $y \in Y$.
However, because we can only evaluate $f$ on $X$ (the observable set) the value of $f$ on the subset $Y \setminus X$ is indeterminable.
As a result, we say that $X$ is \textbf{incomplete} when measuring $f$ on $Y$ if $Y$ is not contained in its support, i.e., $Y \not\subseteq X$.\footnote{%
An equivalent, more rigorous, definition of incompleteness can be given in terms of measure theory: we say that $X$ is \textit{incomplete} when measuring $f$ on $p(Y)$ if $p(Y \setminus X) > 0$.
}

\paragraph{Finite incompleteness.}
We say that $X$ is finitely incomplete when measuring $f$ on $Y$ if the cardinality of the universe, $|\sX|$, is finite.
As an example, consider the setting of KBP:\@
Here, $\sX$ consists of relation triples defined by spans in the document corpus, $X$ is a static test collection, $Y$ is a system's predictions that we are trying to evaluate and $f$ is precision or recall.
While there can be a very large number of such triples, the number is finite because the document collection is finite.
With problems of finite incompleteness, it is fundamentally possible to annotate all of $\sX$ and thus ensure that $Y \subseteq X = \sX$.

\paragraph{Infinite incompleteness.}
Likewise, we say that $X$ is infinitely incomplete when measuring $f$ on $Y$ if the cardinality of the universe, $|\sX|$, is infinite.
As an example, consider the setting of text summarization where $\sX$ consists of all possible text strings which is infinite, $X$ is the set of reference summaries, $Y$ is the set of system-generated summaries and $f$ is the quality of the summary.

\paragraph{Addressing incompleteness with on-demand annotation.}
The main solution for incompleteness we present in this thesis is on-demand human annotation.
Following the formal description above, on-demand annotation ``cheats'' by annotating elements of $\sX$ as required, effectively making all of $\sX$ observable and thus stepping around the problem of incompleteness.
The main issue then is to reduce the costs of measuring $\mu_f$, and this will be the objective of the technical solutions we present through the rest of this thesis.
