Knowledge base population (KBP) systems take in a large document corpus and extract entities and their relations.
Thus far, KBP evaluation has relied on judgements on the pooled predictions of existing systems because it is prohibitively expensive to exhaustively annotate the entire document corpus.
Cast in the framework of finite and infinite completeness we introduced in \refchap{intro}, KBP is a perfect exemplar of finite incompleteness.
In this chapter, we show that this evaluation is problematic:
  when a new system predicts a previously unseen relation, it is penalized even if it is correct.
This leads to significant bias against new systems, which counterproductively discourages innovation in the field.

We then introduce our approach to correcting the bias:
Our first contribution is a new importance-sampling based evaluation which corrects for this bias by annotating a new system's predictions on-demand via crowdsourcing.
We show this eliminates bias and reduces variance using data from the 2015 TAC KBP task. % as compared to standard evaluation.
Our second contribution is an implementation of our method made publicly available as an online KBP evaluation service.
We pilot the service by testing diverse state-of-the-art systems on the TAC KBP 2016 corpus and obtain accurate scores in a cost effective manner. 
