\begin{figure*}[t]
  \centering
  \subcaptionbox{\label{fig:interfaces-edit}Interface to evaluate language quality on CNN/Daily Mail}[0.47\linewidth]{%
    \includegraphics[width=0.47\textwidth]{figures/edit.png}
  }\hfill
  \subcaptionbox{\label{fig:interfaces-qa}Interface to judge answer correctness on MS MARCO}[0.47\linewidth]{%
    \includegraphics[width=0.47\textwidth]{figures/qa.png}
  }
  \caption[Annotation interfaces]{\label{fig:tasks} Screenshots of the annotation interfaces we used to measure (a) summary language quality on CNN/Daily Mail and (b) answer correctness on MS MARCO tasks.
  %Detailed descriptions of the interfaces, instructions and screenshots can be found in \refapp{interfaces}.}
  }
\end{figure*}

\begin{table}[t]
  \centering
  \begin{figure*}[t]
  \centering
  \subcaptionbox{\label{fig:interfaces-edit}Interface to evaluate language quality on CNN/Daily Mail}[0.47\linewidth]{%
    \includegraphics[width=0.47\textwidth]{figures/edit.png}
  }\hfill
  \subcaptionbox{\label{fig:interfaces-qa}Interface to judge answer correctness on MS MARCO}[0.47\linewidth]{%
    \includegraphics[width=0.47\textwidth]{figures/qa.png}
  }
  \caption[Annotation interfaces]{\label{fig:tasks} Screenshots of the annotation interfaces we used to measure (a) summary language quality on CNN/Daily Mail and (b) answer correctness on MS MARCO tasks.
  %Detailed descriptions of the interfaces, instructions and screenshots can be found in \refapp{interfaces}.}
  }
\end{figure*}

\begin{table}[t]
  \centering
  \begin{figure*}[t]
  \centering
  \subcaptionbox{\label{fig:interfaces-edit}Interface to evaluate language quality on CNN/Daily Mail}[0.47\linewidth]{%
    \includegraphics[width=0.47\textwidth]{figures/edit.png}
  }\hfill
  \subcaptionbox{\label{fig:interfaces-qa}Interface to judge answer correctness on MS MARCO}[0.47\linewidth]{%
    \includegraphics[width=0.47\textwidth]{figures/qa.png}
  }
  \caption[Annotation interfaces]{\label{fig:tasks} Screenshots of the annotation interfaces we used to measure (a) summary language quality on CNN/Daily Mail and (b) answer correctness on MS MARCO tasks.
  %Detailed descriptions of the interfaces, instructions and screenshots can be found in \refapp{interfaces}.}
  }
\end{figure*}

\begin{table}[t]
  \centering
  \begin{figure*}[t]
  \centering
  \subcaptionbox{\label{fig:interfaces-edit}Interface to evaluate language quality on CNN/Daily Mail}[0.47\linewidth]{%
    \includegraphics[width=0.47\textwidth]{figures/edit.png}
  }\hfill
  \subcaptionbox{\label{fig:interfaces-qa}Interface to judge answer correctness on MS MARCO}[0.47\linewidth]{%
    \includegraphics[width=0.47\textwidth]{figures/qa.png}
  }
  \caption[Annotation interfaces]{\label{fig:tasks} Screenshots of the annotation interfaces we used to measure (a) summary language quality on CNN/Daily Mail and (b) answer correctness on MS MARCO tasks.
  %Detailed descriptions of the interfaces, instructions and screenshots can be found in \refapp{interfaces}.}
  }
\end{figure*}

\begin{table}[t]
  \centering
  \input{tasks.table}
  \caption[Key statistics of the data collected]{\label{tab:dataset} A summary of the key statistics, human metric variance ($\sigma^2_f$) and annotator variance ($\sigma^2_a$) for different datasets, CNN/Daily Mail (CDM) and MS MARCO in our evaluation benchmark.
  We observe that the relative variance ($\gamma$) is fairly high for most evaluation prompts, upper bounding the data efficiency on these tasks.
  A notable exception is the \texttt{Edit} prompt wherein systems are compared on the number of post-edits required to improve their quality.
  }
\end{table}

%\section{\label{sec:tasks}A generation evaluation benchmark}
\section{\label{sec:tasks} Tasks and datasets}

% == 0. Introduce section: we collected data on a couple of tasks.
In order to compare different approaches to evaluating systems, we first collected human judgments for the output of several automatic summarization and open-response question answering systems using Amazon Mechanical Turk.
Details of instructions provided and quality assurance steps taken are provided in \refapp{interfaces} of the supplementary material.
In this section, we'll briefly describe how we collected this data.
%       - All annotations collected using Amazon Mechanical Turk. Details of instructions provided and quality assurance steps taken in appendix.
%Details of the instructions provided and quality assurance steps taken will be provided in the supplementary material (\refapp{interfaces}).

% == 1. Language quality in automatic summarization
\paragraph{Evaluating language quality in automatic summarization.}
% ++ Task
% - Focus on language quality; content selection important but future work.
% - Summaries from the CNN/Daily Mail dataset.
In automatic summarization, systems must generate a short (on average two or three sentence) summary of an article: for our study, we chose articles from the CNN/Daily Mail (CDM) dataset~\citep{hermann2015read,nallapati2016abstractive} which come paired with reference summaries in the form of story highlights.
%Good summaries must not only pick out the most important information to summarize, but also present this information with fluent text that succinctly recounts the story of the article.
We focus on the \textit{language quality} of summaries and leave evaluating content selection to future work.

% ++ Interface:
% - summary ratings on a scale of 1-3 for fluency, redundancy and overall from DUC.
For each summary, we collected human judgments on a scale from 1--3 (\reffig{interfaces-edit}) for fluency, (lack of) redundancy, and overall quality of the summary using guidelines from the DUC summarization challenge~\citep{dang2006overview}.
%\footnote{%
% - Pilot study of other questions.
%  The DUC summarization challenge also evaluated referential clarity, focus and coherency which we also studied in our pilot experiments. Results for these metrics can be found in our supplementary material.}
% - Tried 1--5 scale but not useful.
%In pilots, we also tried using a 1--5 scale, but found that annotators were unable to so finely discriminate between the quality of mediocre summaries.
% - Edits
As an alternate human metric, we also asked workers to post-edit the system's summary to improve its quality, similar to the post-editing step in MT evaluations~\citep{snover2006ter}.
% - Costs
Obtaining judgments costs about \$0.15 per summary and this cost rises to about \$0.40 per summary for post-editing.

% ++ Systems
%       - Systems: we collected data from X systems (S0, S1, S2, S3, S4). Their average scores and correlations with BLEU is shown in table. An inter annotator variance.
We collected judgments on the summaries generated by the \texttt{seq2seq} and \texttt{pointer} models of \citet{see2017point}, the \texttt{ml} and \texttt{ml+rl} models of \citet{paulus2018deep}, and the reference summaries.\footnote{%
All system output was obtained from the original authors through private communication.} 
Before presenting the summaries to human annotators, we performed some minimal post-processing: we true-cased and de-tokenized the output of \texttt{seq2seq} and \texttt{pointer} using Stanford CoreNLP~\citep{manning2014stanford} and replaced ``unknown'' tokens in each system with a special symbol ($\blacksquare$).

% ++ Stats
%       - Found reference summaries actually fare poorly on fluency because they were sourced from highlights which tend to be fragmented in style.
%Additionally, we also include an existing dataset from \citet{} comprising of language quality ratings for several round-trip machine translation systems~\reffig{tasks}.
%\ac{Include the following statistics: (a) automatic metric correlation for different systems, (b) a graph of a combination of systems versus automatic metric score.}

% == 2. Question answering
\paragraph{Evaluating answer correctness.}
% ++ Task
Next, we look at evaluating the correctness of system outputs in question answering using the MS MARCO question answering dataset~\citep{nguyen2016ms}.
Here, each system is provided with a question and up to 10 paragraphs of context.
The system generates open-response answers that do not need to be tied to a span in any paragraph.

% ++ Interface
We first ask annotators to judge if the output is even plausible for the question,
and if yes,
ask them identify if it is correct according to each context paragraph. 
% - highlights
We found that requiring annotators to highlight regions in the text that support their decision
%(whether or not the answer is correct)
substantially improved the quality of the output without increasing costs.
% - costs
Annotations cost \$0.40 per system response.\footnote{%
  This cost could be significantly reduced if systems also specify which passage they used to generate the answer.
}

% - metrics
While our goal is to evaluate the correctness of the provided answer, we found that there are often answers which may be correct or incorrect depending on the context.
For example, the question ``what is a pothole'' is typically understood to refer to a hole in a roadway, but also refers to a geological feature (\reffig{interfaces-qa}).
This is reflected when annotators mark one context paragraph to support the given answer but mark another to contradict it.
We evaluated systems based on both the average correctness (AvgCorrect) of their answers across all paragraphs
as well as whether their answer is correct according to any paragraph (AnyCorrect).

% ++ Systems
%       - Systems came from FastQA, S-NET.
We collected annotations on the systems generated by the \texttt{fastqa} and
\texttt{fastqa\_ext} from \citet{weissenborn2017fastqa} and the \texttt{snet} and \texttt{snet.ens}(emble) models from \citet{tan2018s}, along with reference answers.
The answers generated by the systems were used without any post-processing.
Surprisingly, we found that the correctness of the reference answers (according to the AnyCorrect metric) was only 73.5\%,
only 2\% above that of the leading system ($\texttt{snet.ens}$).
We manually inspected 30 reference answers which were annotated incorrectly and found that of those, 
about 95\% were indeed incorrect.
However, 62\% are actually answerable from some paragraph,
indicating that the real ceiling performance on this dataset is around 90\% and
that there is still room for improvement on this task.

  \caption[Key statistics of the data collected]{\label{tab:dataset} A summary of the key statistics, human metric variance ($\sigma^2_f$) and annotator variance ($\sigma^2_a$) for different datasets, CNN/Daily Mail (CDM) and MS MARCO in our evaluation benchmark.
  We observe that the relative variance ($\gamma$) is fairly high for most evaluation prompts, upper bounding the data efficiency on these tasks.
  A notable exception is the \texttt{Edit} prompt wherein systems are compared on the number of post-edits required to improve their quality.
  }
\end{table}

%\section{\label{sec:tasks}A generation evaluation benchmark}
\section{\label{sec:tasks} Tasks and datasets}

% == 0. Introduce section: we collected data on a couple of tasks.
In order to compare different approaches to evaluating systems, we first collected human judgments for the output of several automatic summarization and open-response question answering systems using Amazon Mechanical Turk.
Details of instructions provided and quality assurance steps taken are provided in \refapp{interfaces} of the supplementary material.
In this section, we'll briefly describe how we collected this data.
%       - All annotations collected using Amazon Mechanical Turk. Details of instructions provided and quality assurance steps taken in appendix.
%Details of the instructions provided and quality assurance steps taken will be provided in the supplementary material (\refapp{interfaces}).

% == 1. Language quality in automatic summarization
\paragraph{Evaluating language quality in automatic summarization.}
% ++ Task
% - Focus on language quality; content selection important but future work.
% - Summaries from the CNN/Daily Mail dataset.
In automatic summarization, systems must generate a short (on average two or three sentence) summary of an article: for our study, we chose articles from the CNN/Daily Mail (CDM) dataset~\citep{hermann2015read,nallapati2016abstractive} which come paired with reference summaries in the form of story highlights.
%Good summaries must not only pick out the most important information to summarize, but also present this information with fluent text that succinctly recounts the story of the article.
We focus on the \textit{language quality} of summaries and leave evaluating content selection to future work.

% ++ Interface:
% - summary ratings on a scale of 1-3 for fluency, redundancy and overall from DUC.
For each summary, we collected human judgments on a scale from 1--3 (\reffig{interfaces-edit}) for fluency, (lack of) redundancy, and overall quality of the summary using guidelines from the DUC summarization challenge~\citep{dang2006overview}.
%\footnote{%
% - Pilot study of other questions.
%  The DUC summarization challenge also evaluated referential clarity, focus and coherency which we also studied in our pilot experiments. Results for these metrics can be found in our supplementary material.}
% - Tried 1--5 scale but not useful.
%In pilots, we also tried using a 1--5 scale, but found that annotators were unable to so finely discriminate between the quality of mediocre summaries.
% - Edits
As an alternate human metric, we also asked workers to post-edit the system's summary to improve its quality, similar to the post-editing step in MT evaluations~\citep{snover2006ter}.
% - Costs
Obtaining judgments costs about \$0.15 per summary and this cost rises to about \$0.40 per summary for post-editing.

% ++ Systems
%       - Systems: we collected data from X systems (S0, S1, S2, S3, S4). Their average scores and correlations with BLEU is shown in table. An inter annotator variance.
We collected judgments on the summaries generated by the \texttt{seq2seq} and \texttt{pointer} models of \citet{see2017point}, the \texttt{ml} and \texttt{ml+rl} models of \citet{paulus2018deep}, and the reference summaries.\footnote{%
All system output was obtained from the original authors through private communication.} 
Before presenting the summaries to human annotators, we performed some minimal post-processing: we true-cased and de-tokenized the output of \texttt{seq2seq} and \texttt{pointer} using Stanford CoreNLP~\citep{manning2014stanford} and replaced ``unknown'' tokens in each system with a special symbol ($\blacksquare$).

% ++ Stats
%       - Found reference summaries actually fare poorly on fluency because they were sourced from highlights which tend to be fragmented in style.
%Additionally, we also include an existing dataset from \citet{} comprising of language quality ratings for several round-trip machine translation systems~\reffig{tasks}.
%\ac{Include the following statistics: (a) automatic metric correlation for different systems, (b) a graph of a combination of systems versus automatic metric score.}

% == 2. Question answering
\paragraph{Evaluating answer correctness.}
% ++ Task
Next, we look at evaluating the correctness of system outputs in question answering using the MS MARCO question answering dataset~\citep{nguyen2016ms}.
Here, each system is provided with a question and up to 10 paragraphs of context.
The system generates open-response answers that do not need to be tied to a span in any paragraph.

% ++ Interface
We first ask annotators to judge if the output is even plausible for the question,
and if yes,
ask them identify if it is correct according to each context paragraph. 
% - highlights
We found that requiring annotators to highlight regions in the text that support their decision
%(whether or not the answer is correct)
substantially improved the quality of the output without increasing costs.
% - costs
Annotations cost \$0.40 per system response.\footnote{%
  This cost could be significantly reduced if systems also specify which passage they used to generate the answer.
}

% - metrics
While our goal is to evaluate the correctness of the provided answer, we found that there are often answers which may be correct or incorrect depending on the context.
For example, the question ``what is a pothole'' is typically understood to refer to a hole in a roadway, but also refers to a geological feature (\reffig{interfaces-qa}).
This is reflected when annotators mark one context paragraph to support the given answer but mark another to contradict it.
We evaluated systems based on both the average correctness (AvgCorrect) of their answers across all paragraphs
as well as whether their answer is correct according to any paragraph (AnyCorrect).

% ++ Systems
%       - Systems came from FastQA, S-NET.
We collected annotations on the systems generated by the \texttt{fastqa} and
\texttt{fastqa\_ext} from \citet{weissenborn2017fastqa} and the \texttt{snet} and \texttt{snet.ens}(emble) models from \citet{tan2018s}, along with reference answers.
The answers generated by the systems were used without any post-processing.
Surprisingly, we found that the correctness of the reference answers (according to the AnyCorrect metric) was only 73.5\%,
only 2\% above that of the leading system ($\texttt{snet.ens}$).
We manually inspected 30 reference answers which were annotated incorrectly and found that of those, 
about 95\% were indeed incorrect.
However, 62\% are actually answerable from some paragraph,
indicating that the real ceiling performance on this dataset is around 90\% and
that there is still room for improvement on this task.

  \caption[Key statistics of the data collected]{\label{tab:dataset} A summary of the key statistics, human metric variance ($\sigma^2_f$) and annotator variance ($\sigma^2_a$) for different datasets, CNN/Daily Mail (CDM) and MS MARCO in our evaluation benchmark.
  We observe that the relative variance ($\gamma$) is fairly high for most evaluation prompts, upper bounding the data efficiency on these tasks.
  A notable exception is the \texttt{Edit} prompt wherein systems are compared on the number of post-edits required to improve their quality.
  }
\end{table}

%\section{\label{sec:tasks}A generation evaluation benchmark}
\section{\label{sec:tasks} Tasks and datasets}

% == 0. Introduce section: we collected data on a couple of tasks.
In order to compare different approaches to evaluating systems, we first collected human judgments for the output of several automatic summarization and open-response question answering systems using Amazon Mechanical Turk.
Details of instructions provided and quality assurance steps taken are provided in \refapp{interfaces} of the supplementary material.
In this section, we'll briefly describe how we collected this data.
%       - All annotations collected using Amazon Mechanical Turk. Details of instructions provided and quality assurance steps taken in appendix.
%Details of the instructions provided and quality assurance steps taken will be provided in the supplementary material (\refapp{interfaces}).

% == 1. Language quality in automatic summarization
\paragraph{Evaluating language quality in automatic summarization.}
% ++ Task
% - Focus on language quality; content selection important but future work.
% - Summaries from the CNN/Daily Mail dataset.
In automatic summarization, systems must generate a short (on average two or three sentence) summary of an article: for our study, we chose articles from the CNN/Daily Mail (CDM) dataset~\citep{hermann2015read,nallapati2016abstractive} which come paired with reference summaries in the form of story highlights.
%Good summaries must not only pick out the most important information to summarize, but also present this information with fluent text that succinctly recounts the story of the article.
We focus on the \textit{language quality} of summaries and leave evaluating content selection to future work.

% ++ Interface:
% - summary ratings on a scale of 1-3 for fluency, redundancy and overall from DUC.
For each summary, we collected human judgments on a scale from 1--3 (\reffig{interfaces-edit}) for fluency, (lack of) redundancy, and overall quality of the summary using guidelines from the DUC summarization challenge~\citep{dang2006overview}.
%\footnote{%
% - Pilot study of other questions.
%  The DUC summarization challenge also evaluated referential clarity, focus and coherency which we also studied in our pilot experiments. Results for these metrics can be found in our supplementary material.}
% - Tried 1--5 scale but not useful.
%In pilots, we also tried using a 1--5 scale, but found that annotators were unable to so finely discriminate between the quality of mediocre summaries.
% - Edits
As an alternate human metric, we also asked workers to post-edit the system's summary to improve its quality, similar to the post-editing step in MT evaluations~\citep{snover2006ter}.
% - Costs
Obtaining judgments costs about \$0.15 per summary and this cost rises to about \$0.40 per summary for post-editing.

% ++ Systems
%       - Systems: we collected data from X systems (S0, S1, S2, S3, S4). Their average scores and correlations with BLEU is shown in table. An inter annotator variance.
We collected judgments on the summaries generated by the \texttt{seq2seq} and \texttt{pointer} models of \citet{see2017point}, the \texttt{ml} and \texttt{ml+rl} models of \citet{paulus2018deep}, and the reference summaries.\footnote{%
All system output was obtained from the original authors through private communication.} 
Before presenting the summaries to human annotators, we performed some minimal post-processing: we true-cased and de-tokenized the output of \texttt{seq2seq} and \texttt{pointer} using Stanford CoreNLP~\citep{manning2014stanford} and replaced ``unknown'' tokens in each system with a special symbol ($\blacksquare$).

% ++ Stats
%       - Found reference summaries actually fare poorly on fluency because they were sourced from highlights which tend to be fragmented in style.
%Additionally, we also include an existing dataset from \citet{} comprising of language quality ratings for several round-trip machine translation systems~\reffig{tasks}.
%\ac{Include the following statistics: (a) automatic metric correlation for different systems, (b) a graph of a combination of systems versus automatic metric score.}

% == 2. Question answering
\paragraph{Evaluating answer correctness.}
% ++ Task
Next, we look at evaluating the correctness of system outputs in question answering using the MS MARCO question answering dataset~\citep{nguyen2016ms}.
Here, each system is provided with a question and up to 10 paragraphs of context.
The system generates open-response answers that do not need to be tied to a span in any paragraph.

% ++ Interface
We first ask annotators to judge if the output is even plausible for the question,
and if yes,
ask them identify if it is correct according to each context paragraph. 
% - highlights
We found that requiring annotators to highlight regions in the text that support their decision
%(whether or not the answer is correct)
substantially improved the quality of the output without increasing costs.
% - costs
Annotations cost \$0.40 per system response.\footnote{%
  This cost could be significantly reduced if systems also specify which passage they used to generate the answer.
}

% - metrics
While our goal is to evaluate the correctness of the provided answer, we found that there are often answers which may be correct or incorrect depending on the context.
For example, the question ``what is a pothole'' is typically understood to refer to a hole in a roadway, but also refers to a geological feature (\reffig{interfaces-qa}).
This is reflected when annotators mark one context paragraph to support the given answer but mark another to contradict it.
We evaluated systems based on both the average correctness (AvgCorrect) of their answers across all paragraphs
as well as whether their answer is correct according to any paragraph (AnyCorrect).

% ++ Systems
%       - Systems came from FastQA, S-NET.
We collected annotations on the systems generated by the \texttt{fastqa} and
\texttt{fastqa\_ext} from \citet{weissenborn2017fastqa} and the \texttt{snet} and \texttt{snet.ens}(emble) models from \citet{tan2018s}, along with reference answers.
The answers generated by the systems were used without any post-processing.
Surprisingly, we found that the correctness of the reference answers (according to the AnyCorrect metric) was only 73.5\%,
only 2\% above that of the leading system ($\texttt{snet.ens}$).
We manually inspected 30 reference answers which were annotated incorrectly and found that of those, 
about 95\% were indeed incorrect.
However, 62\% are actually answerable from some paragraph,
indicating that the real ceiling performance on this dataset is around 90\% and
that there is still room for improvement on this task.

  \caption[Key statistics of the data collected]{\label{tab:dataset} A summary of the key statistics, human metric variance ($\sigma^2_f$) and annotator variance ($\sigma^2_a$) for different datasets, CNN/Daily Mail (CDM) and MS MARCO in our evaluation benchmark.
  We observe that the relative variance ($\gamma$) is fairly high for most evaluation prompts, upper bounding the data efficiency on these tasks.
  A notable exception is the \texttt{Edit} prompt wherein systems are compared on the number of post-edits required to improve their quality.
  }
\end{table}

%\section{\label{sec:tasks}A generation evaluation benchmark}
\section{\label{sec:tasks} Tasks and datasets}

% == 0. Introduce section: we collected data on a couple of tasks.
In order to compare different approaches to evaluating systems, we first collected human judgments for the output of several automatic summarization and open-response question answering systems using Amazon Mechanical Turk.
Details of instructions provided and quality assurance steps taken are provided in \refapp{interfaces} of the supplementary material.
In this section, we'll briefly describe how we collected this data.
%       - All annotations collected using Amazon Mechanical Turk. Details of instructions provided and quality assurance steps taken in appendix.
%Details of the instructions provided and quality assurance steps taken will be provided in the supplementary material (\refapp{interfaces}).

% == 1. Language quality in automatic summarization
\paragraph{Evaluating language quality in automatic summarization.}
% ++ Task
% - Focus on language quality; content selection important but future work.
% - Summaries from the CNN/Daily Mail dataset.
In automatic summarization, systems must generate a short (on average two or three sentence) summary of an article: for our study, we chose articles from the CNN/Daily Mail (CDM) dataset~\citep{hermann2015read,nallapati2016abstractive} which come paired with reference summaries in the form of story highlights.
%Good summaries must not only pick out the most important information to summarize, but also present this information with fluent text that succinctly recounts the story of the article.
We focus on the \textit{language quality} of summaries and leave evaluating content selection to future work.

% ++ Interface:
% - summary ratings on a scale of 1-3 for fluency, redundancy and overall from DUC.
For each summary, we collected human judgments on a scale from 1--3 (\reffig{interfaces-edit}) for fluency, (lack of) redundancy, and overall quality of the summary using guidelines from the DUC summarization challenge~\citep{dang2006overview}.
%\footnote{%
% - Pilot study of other questions.
%  The DUC summarization challenge also evaluated referential clarity, focus and coherency which we also studied in our pilot experiments. Results for these metrics can be found in our supplementary material.}
% - Tried 1--5 scale but not useful.
%In pilots, we also tried using a 1--5 scale, but found that annotators were unable to so finely discriminate between the quality of mediocre summaries.
% - Edits
As an alternate human metric, we also asked workers to post-edit the system's summary to improve its quality, similar to the post-editing step in MT evaluations~\citep{snover2006ter}.
% - Costs
Obtaining judgments costs about \$0.15 per summary and this cost rises to about \$0.40 per summary for post-editing.

% ++ Systems
%       - Systems: we collected data from X systems (S0, S1, S2, S3, S4). Their average scores and correlations with BLEU is shown in table. An inter annotator variance.
We collected judgments on the summaries generated by the \texttt{seq2seq} and \texttt{pointer} models of \citet{see2017point}, the \texttt{ml} and \texttt{ml+rl} models of \citet{paulus2018deep}, and the reference summaries.\footnote{%
All system output was obtained from the original authors through private communication.} 
Before presenting the summaries to human annotators, we performed some minimal post-processing: we true-cased and de-tokenized the output of \texttt{seq2seq} and \texttt{pointer} using Stanford CoreNLP~\citep{manning2014stanford} and replaced ``unknown'' tokens in each system with a special symbol ($\blacksquare$).

% ++ Stats
%       - Found reference summaries actually fare poorly on fluency because they were sourced from highlights which tend to be fragmented in style.
%Additionally, we also include an existing dataset from \citet{} comprising of language quality ratings for several round-trip machine translation systems~\reffig{tasks}.
%\ac{Include the following statistics: (a) automatic metric correlation for different systems, (b) a graph of a combination of systems versus automatic metric score.}

% == 2. Question answering
\paragraph{Evaluating answer correctness.}
% ++ Task
Next, we look at evaluating the correctness of system outputs in question answering using the MS MARCO question answering dataset~\citep{nguyen2016ms}.
Here, each system is provided with a question and up to 10 paragraphs of context.
The system generates open-response answers that do not need to be tied to a span in any paragraph.

% ++ Interface
We first ask annotators to judge if the output is even plausible for the question,
and if yes,
ask them identify if it is correct according to each context paragraph. 
% - highlights
We found that requiring annotators to highlight regions in the text that support their decision
%(whether or not the answer is correct)
substantially improved the quality of the output without increasing costs.
% - costs
Annotations cost \$0.40 per system response.\footnote{%
  This cost could be significantly reduced if systems also specify which passage they used to generate the answer.
}

% - metrics
While our goal is to evaluate the correctness of the provided answer, we found that there are often answers which may be correct or incorrect depending on the context.
For example, the question ``what is a pothole'' is typically understood to refer to a hole in a roadway, but also refers to a geological feature (\reffig{interfaces-qa}).
This is reflected when annotators mark one context paragraph to support the given answer but mark another to contradict it.
We evaluated systems based on both the average correctness (AvgCorrect) of their answers across all paragraphs
as well as whether their answer is correct according to any paragraph (AnyCorrect).

% ++ Systems
%       - Systems came from FastQA, S-NET.
We collected annotations on the systems generated by the \texttt{fastqa} and
\texttt{fastqa\_ext} from \citet{weissenborn2017fastqa} and the \texttt{snet} and \texttt{snet.ens}(emble) models from \citet{tan2018s}, along with reference answers.
The answers generated by the systems were used without any post-processing.
Surprisingly, we found that the correctness of the reference answers (according to the AnyCorrect metric) was only 73.5\%,
only 2\% above that of the leading system ($\texttt{snet.ens}$).
We manually inspected 30 reference answers which were annotated incorrectly and found that of those, 
about 95\% were indeed incorrect.
However, 62\% are actually answerable from some paragraph,
indicating that the real ceiling performance on this dataset is around 90\% and
that there is still room for improvement on this task.
