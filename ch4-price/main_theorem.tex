\subsection{Main Theorem}

In this section, we prove the main theorem (\refthm{price:main}) in the \refchap{price} about the minimax optimal variance for an unbiased estimator. \refthm{price:main} will follow from the two following lemmas (Lemmas~\ref{lem:variance_calc} and~\ref{lem:mvue}). First, we show in \reflem{variance_calc} that for all distributions with fixed $\sigma^2_f$, $\sigma^2_a$ and $\rho$, the variance of $\mucontrol$ is constant and equal to: $\frac{1}{n}(\sigma_f^2(1-\rho^2)+\sigma_a^2)$.
Then we give an explicit distribution, a Gaussian distribution, where \emph{any} estimator yields at least this variance using the theory of sufficient statistics.
Together, these show that the max variance of any estimator is at least the max variance of $\mucontrol$.

As a reminder, the estimator is 

\begin{align}
\mucontrol = \frac{1}{n} \sum_i y^{(i)} - \alpha g(z^{(i)})
\end{align}

where $\alpha = \Cov(f(z),g(z))$.

%%%%%%%%%%%%%%%%%%%%%%%%%
\begin{lemma}
\label{lem:variance_calc}
The variance of $\mucontrol$ is always

\begin{align}
\frac{1}{n}(\sigma^2_f (1-\rho^2) + \sigma^2_a)
\end{align}

\end{lemma}
\begin{proof}
By the law of total variance, with respect to the draws of $z^{(i)}$,

\begin{align}
\Var(\mucontrol) = \mathbb{E}_{z^{(i)}}[\Var(\mucontrol | z^{(i)})] + \Var_{z^{(i)}} ( \mathbb{E}[\mucontrol | z^{(i)} ] ) 
\end{align}

We will evaluate each of the two terms on the right hand side. 

For the first term,

\begin{align}
\mathbb{E}_{z^{(i)}}[\Var(\mucontrol | z^{(i)})] = \mathbb{E}_{z^{(i)}} \left[ \Var \left( \frac{1}{n} \sum_i y^{(i)} | z^{(i)} \right) \right]
\end{align}
Because the human responses $Y(z^{(i)})$ are uncorrelated,

\begin{align}
\mathbb{E}_{z^{(i)}}[\Var(\mucontrol | z^{(i)})] &= \mathbb{E}_{z^{(i)}} \left[ \frac{1}{n^2} \sum_i \Var( Y(z^{(i)}) ) | z^{(i)} \right] \\
&= \frac{1}{n} \mathbb{E}_z [\Var(Y(z))] \\
&= \frac{1}{n} \sigma^2_a
\end{align}

For the second term,

\begin{align}
\Var_{z^{(i)}}( \mathbb{E}[\mucontrol | z^{(i)}]) = \Var_{z^{(i)}} \left( \frac{1}{n} \sum_i f(z^{(i)}) - \alpha g(z^{(i)}) \right)
\end{align}

Because the $z^{(i)}$ are sampled independently,

\begin{align}
\Var_{z^{(i)}}( \mathbb{E}[\mucontrol | z^{(i)}]) &= \frac{1}{n} \Var( f(z) - \alpha g(z)) \\
&= \frac{1}{n} [\Var(f(z)) - 2 \alpha \Cov(f(z),g(z)) + \alpha^2 \Var(g(z))]
\end{align}

Note that $\Var(f(z))=\sigma^2_f$, $\Cov(f(z),g(z))=\alpha$, and $\Var(g(z))=1$ (since it is normalized). Thus,

\begin{align}
\Var_{z^{(i)}}( \mathbb{E}[\mucontrol | z^{(i)}]) &= \frac{1}{n} [\sigma^2_f - 2 \alpha^2 + \alpha^2] \\
&=\frac{1}{n} [\sigma^2_f - \alpha^2]
\end{align}

Since the correlation $\rho = \frac{\alpha}{\sigma_f \sigma_g} = \frac{\alpha}{\sigma_f}$,

\begin{align}
\Var_{z^{(i)}}( \mathbb{E}[\mucontrol | z^{(i)}]) &= \frac{1}{n} [\sigma^2_f - \sigma^2_f \rho^2] \\
&=\frac{1}{n} \sigma^2_f (1-\rho^2)
\end{align}

Putting these two terms together, we find that,

\begin{align}
\Var(\mucontrol) &= \frac{1}{n} \sigma^2_a + \frac{1}{n} \sigma^2_f (1-\rho^2) \\
&= \frac{1}{n}(\sigma^2_f (1 - \rho^2) + \sigma^2_a)
\end{align}
\end{proof}
%%%%%%%%%%%%%%%%%%%%%%%%%%%%%%%%

For the next lemma, we show that the worst-case variance for any estimator is at least that of $\mucontrol$. For this, we will define a simple Gaussian distribution and use the theory of sufficient statistics. We explicitly define a distribution over $f(z)$, $g(z)$, and $Y(Z) - f(z)$. In particular, we assume these are all Gaussian distributions with respective means, $\mu, 0, 0$,  and variances, $\sigma^2_f, 1, \sigma^2_a$. Additionally, we assume that $f(z)$ and $g(z)$ have covariance $\alpha$ but $Y(z) - f(z)$ is independent.

%%%%%%%%%%%%%%%%%%%%%%%%%%%%%%
\begin{lemma}
\label{lem:mvue}
$\mucontrol$ is the minimal variance unbiased estimate (MVUE) for the Gaussian distribution above.
\end{lemma}
\begin{proof}
The proof is straightforward: we first show that $\mucontrol$ is a sufficient statistic using the Fisher-Neyman factorization theorem, and then we apply the Lehman-Scheffe theorem.

For ease of notation, define $g_i = g(z^{(i)})$ and $y_i = y^{(i)}$. For the purposes of statistics, only $\mu$ is a parameter; the other ``parameters'' are known constants. %Let $c_2 = \sigma^2_f + \sigma^2_a$ be a named constant. 
Note that the pdf of the observed variables $g_i$ and $y_i$ is,

\begin{align}
\prod_i c_1 \exp(-\frac{1}{2} 
\begin{bmatrix}
(y_i - \mu) \\
g_i \\
\end{bmatrix}^T
\begin{bmatrix}
\sigma^2_f + \sigma^2_a & \alpha \\
\alpha & 1 \\
\end{bmatrix}^{-1}
\begin{bmatrix}
(y_i - \mu) \\
g_i \\
\end{bmatrix})
\end{align}

\begin{align}
=c_2 \exp(-\frac{1}{2} \sum_i
\begin{bmatrix}
(y_i - \mu) \\
g_i \\
\end{bmatrix}^T
\begin{bmatrix}
\sigma^2_f + \sigma^2_a & \alpha \\
\alpha & 1 \\
\end{bmatrix}^{-1}
\begin{bmatrix}
(y_i - \mu) \\
g_i \\
\end{bmatrix})
\end{align}

Thus, with the Fisher-Neyman factorization theorem, it suffices to show that the exponetiated term $T$ decomposes as a sum of a function that only depends on the data and a function that only depends on $\mucontrol$ and $\mu$.

\begin{align}
  T &=
\sum_i
\begin{bmatrix}
(y_i - \mu) \\
g_i \\
\end{bmatrix}^T
\begin{bmatrix}
\sigma^2_f + \sigma^2_a & \alpha \\
\alpha & 1 \\
\end{bmatrix}^{-1}
\begin{bmatrix}
(y_i - \mu) \\
g_i \\
\end{bmatrix}
\end{align}

Letting $c_3$ be the inverse determinant (which is constant),

\begin{align}
T &= c_3 \sum_i
\begin{bmatrix}
(y_i - \mu) \\
g_i \\
\end{bmatrix}^T
\begin{bmatrix}
1 & -\alpha \\
-\alpha & \sigma^2_f + \sigma^2_a \\
\end{bmatrix}
\begin{bmatrix}
(y_i - \mu) \\
g_i \\
\end{bmatrix} \\
&= c_3 \left[ \sum_i (y_i - \mu)^2 - 2 \alpha \sum_i (y_i - \mu) g_i + (\sigma^2_f + \sigma^2_a) \sum_i g_i^2 \right] \\
&= c_3 \left[ \sum_i y_i^2 - 2 \mu \sum_i y_i + n \mu^2 - 2 \alpha \sum_i y_i g_i + 2 \alpha \mu \sum_i g_i  + (\sigma^2_f + \sigma^2_a) \sum_i g_i^2 \right] \\
&=  -2c_3 \mu \left[\sum_i y_i -  \alpha \sum_i g_i \right]  + c_3n\mu^2 +  c_3 \left[ \sum_i y_i^2 - 2 \alpha \sum_i y_i g_i + (\sigma^2_f + \sigma^2_a) \sum_i g_i^2 \right] \\
&=  -2 n c_3 \mu \mucontrol  +  c_3n\mu^2 + c_3\left[ \sum_i y_i^2 - 2 \alpha \sum_i y_i g_i + (\sigma^2_f + \sigma^2_a) \sum_i g_i^2 \right]
\end{align}

Thus, we see the decomposition into the function of only the data on the right and only $\mu$ and $\mucontrol$ on the left. Thus, $\mucontrol$ is a sufficient statistic.

Further, $\mucontrol$ is an unbiased estimate of $\mu$ since $\mathbb{E}[g_i]=0$ and $\mathbb{E}[y_i] = \mu$. 

Further, since $\mucontrol$ is normally distributed with mean dependent on $\mu$, it is complete. 

Thus, by the Lehmann-Scheffe theorem, $\mucontrol$ is the minimal variance unbiased estimate (MVUE).

\end{proof}

\begin{thm} [\ref{thm:price:main}]
Among all unbiased
  estimators that are functions of $y^{(i)}$ and $g(z^{(i)})$, and for all distributions with a given $\sigma^2_f$, $\sigma^2_a$, and $\alpha$,
\begin{align}
  \Var(\mucontrol) = \frac{1}{n} (\sigma^2_f (1 - \rho^2) + \sigma^2_a),
\end{align}
and no other estimator has a lower worst-case variance.
\end{thm}
\begin{proof}
From \reflem{variance_calc} we have that the max variance of $\mucontrol$ over all distributions with fixed variances, is exactly,

\begin{align}
\frac{1}{n} (\sigma^2_f (1 - \rho^2) + \sigma^2_a)
\end{align}

Further, from \reflem{mvue}, we know that $\mucontrol$ is the MVUE for a particular class of distributions, thus, any estimator has a larger max variance over all distributions. 

Combining these two facts, we get that the minimax variance is the variance of $\mucontrol$.
\end{proof}
